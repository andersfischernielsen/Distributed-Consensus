% !TeX root = ../DCR-Consensus.tex
\section{Produce}
\subsection{Goal} % Hvad er problemet
The goal of the algortithm is, given a DCR graph, to fetch and merge the histories of each event in the graph into one history, such that the resulting history is partially ordered in accordance to the ruleset described in \autoref{chap:domainanalysis}. 

%Correcness should be based on:
%\begin{itemize}
%    \item How many nodes' history gets fetched in the workflow (higher is better).
%    \item How rendundant the data is (higher redundancy is better).
%    \item How well it handles cycles in the graph
%\end{itemize}

\subsection{Implementation} % hvordan løses det
The implementation of the algorithm starts at a single event in the workflow and is called recursively on the reachable events in the workflow. A reachable event in the workflow constitutes any event that has a relation from the current event in scope. \todo[inline]{Udvid beskrivelse til at inkludere løsning hvis event peger på nuværende event. Dette event er p.t. ikke "synligt".} \bigskip

The implemented solution uses two main lists, called \texttt{request trace} and \texttt{wait for}. The current history has been assigned a unique ID, called \texttt{history ID}.

The algorithm propagates throughout the graph using recursion with checks during execution in order to avoid infinite cycles of requests. \bigskip

When an event receives a request for its history, it adds every reachable event from itself to \texttt{wait for}, and sends requests to these events for history. The request for history includes a \texttt{request trace}, which is initially empty. Outgoing history requests from an event include the received \texttt{request trace}, with the ID of the event appended. This enables events to do cycle detection. \bigskip

When an event receives a history request, it checks any event ID in the received \texttt{request trace} exists in its own \texttt{wait for}. Any event ID in \texttt{wait for} that exists in \texttt{request trace} is removed and the history of the event is returned to the requester. 
Cycles are therefore avoided, since a chain of events are unable to wait for any event in the chain. \bigskip 


\subsubsection*{Pseudocode}
\begin{algorithmic}
	\State Event receives request for history with a \texttt{request trace}, $T$ and a \texttt{history id}, $HID$
	\State Event has internal \texttt{wait for}, $W$
	\State
	\State $W\gets W::$ reachable events \Comment Add all reachable events to \texttt{wait for}.
	\If {$W=\emptyset$}
	\Return own history
	\Else
	\State $W\gets W-T$ \Comment Remove every event in $T$ from $W$.
	\State $T\gets T::ownID$ \Comment Append own \texttt{HID} to $T$.
	\State
	\ForAll {$w$ in $W$} 
	\State request history from $w$ with $T$ \Comment Request from neighbours with new \texttt{trace}.
	\EndFor
	\State wait for receiving histories
	\State stitch received histories with own history
	\State
	\Return stitched histories
	\EndIf
\end{algorithmic}

\subsection{Discussion} % bliver problemet løst
\subsection{Performance} % hvor godt løses det