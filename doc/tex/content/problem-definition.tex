% !TeX root = ../DCR-Consensus.tex
\chapter{Problem Definition}

A DCR graph consists of one or more events that each constitute a node in the graph. Relations between nodes are edges in the graph.
It is desired to find the order of execution in a DCR graph. Consensus should therefore be reached between nodes in the graph about the order of execution. Finding this order raises several problems.


Finding a history or log of a given order of execution can be difficult in distributed DCR graphs, due to the fact that no single node has the overview of the entire workflow in the graph.
Logs can be split among several nodes, timestamps of logs do not necessarily correspond among nodes and nodes can emit erroneous logs. These challenges must be overcome in order to reach consensus on the order of execution.
