% !TeX root = ../DCR-Consensus.tex
% !TeX spellcheck = en_GB
\chapter{Domain Analysis}
\label{chap:domainanalysis}

\section{What is a History}
\begin{definition}
	An \textit{\textbf{Action}}\index{Action|(} contains the following information:
	\begin{itemize}
		\item The Id of the source Event, this is the Event that has logged the action.
		\item The time stamp of the action local to the source Event.
		\item The Id of the participating Event, this is the Event that are being contacted, or the Event who has initiated the action.
		\item The type of the Action, i.e. \texttt{IncludedBy}.
	\end{itemize}
	
	The reason that an Action is not called Event\index{Event}, is because this term is used for an Event in the DCR Graph workflow.
	\index{Action|)}
\end{definition}

\begin{definition}\label{definition:historya}
	A \textit{\textbf{History}}\index{History} is a set of Actions.\index{Action}
\end{definition}

\begin{lemma}\label{lemma:partialorder}
	Locally, a history is totally ordered.\todo[inline]{Find et sted hvor total order matematisk defineres - Jeg synes \url{https://en.wikipedia.org/wiki/Total_order} er lidt for slap. - Mikael.} This is because the source Event knows the timestamps of all Actions and can order them by this. 
	
	In a workflow these local histories are put together. This creates the possibility of having histories that cannot be totally ordered, because there haven't been exchanged messages between the source Events between the actions. This means that it is not possible to decide between all pairs of actions, which action happened first.
	
	There is one limitation on the order of the actions that must be fulfilled by any history, locally or globally. That is, that the history cannot contain cycles as this would imply that two (or more) actions would have to happen before the other.
	
	\todo[inline]{Lav et argument, eller find nogle andre der har defineret at en total order også er en partial order. (I forhold til Wiki-definitionen ovenfor, er det er en partial order ikke har den 3. begrænsning, men at det ellers er det samme, og derfor er partial order en svagere definition af total order.)}
	
	This creates the need for a history of partial order.
\end{lemma}

\newpar Because of \autoref{lemma:partialorder} we can update \autoref{definition:historya} to the following definition:

\begin{definition}
	A \textit{\textbf{History}}\index{History} is a \textit{partially ordered} set of Actions.\index{Action}
	\label{definition:historyb}
\end{definition}

\section{Relations between logs}
\label{sec:relations}
All logs have the form (\#Number, FromID, RelationType, ToID/ByID)
\subsection{Inclusion}
\begin{tabularx}{\textwidth}{|*{3}{>{\raggedright\arraybackslash}X|}}
  \hline
  Relation & Event1 Log & Event2 Log \\
  \hline
  Event 1 includes Event 2 & (\#1, Event 1, includes, Event 2) & (\#1, Event 2, includedBy, Event 1) \\
  \hline
  Event 2 includes Event 1 & (\#1, Event 1, includedBy, Event 2) & (\#3, Event 2, includes, Event 1) \\
  \hline
\end{tabularx}

\subsection{Exclusion}

\begin{tabularx}{\textwidth}{|*{3}{>{\raggedright\arraybackslash}X|}}
  \hline
  Relation & Event1 Log & Event2 Log \\
  \hline
  Event 1 excludes Event 2 & (\#1, Event 1, excludes, Event 2) & (\#1, Event 2, excludedBy, Event 1) \\
  \hline
  Event 2 excludes Event 1 & (\#1, Event 1, excludedBy, Event 2) & (\#3, Event 2, excludes, Event 1) \\
  \hline
\end{tabularx}

\subsection{Pending}
\begin{tabularx}{\textwidth}{|*{3}{>{\raggedright\arraybackslash}X|}}
  \hline
  Relation & Event1 Log & Event2 Log \\
  \hline
  Event 1 sets Pending Event 2 & (\#1, Event 1, setsPending, Event 2) & (\#1, Event 2, setPendingBy, Event 1) \\
  \hline
  Event 2 sets Pending Event 1 & (\#1, Event 1, setPendingBy, Event 2) & (\#3, Event 2, setsPending, Event 1) \\
  \hline
\end{tabularx}

\subsection{Conditions}
\begin{tabularx}{\textwidth}{|*{3}{>{\raggedright\arraybackslash}X|}}\hline
	Relation & Event1 Log & Event2 Log \\\hline
	Event 1 checks Event 2 which is executable & (\#1, Event 1, ConditionChecks true, Event 2) & (\#1, Event 2, ConditionChecked true, Event 1) \\\hline
	Event 1 checks Event 2 which is not executable & (\#1, Event 1, ConditionChecks false, Event 2) & (\#3, Event 2, ConditionChecked false, Event 1) \\\hline
	Event 2 checks Event 1 which is executable & (\#1, Event 1, ConditionChecked true, Event 2) & (\#1, Event 2, ConditionChecks true, Event 1) \\\hline
	Event 2 checks Event 1 which is not executable & (\#1, Event 1, ConditionChecked false, Event 2) & (\#3, Event 2, ConditionChecks false, Event 1) \\\hline
\end{tabularx}

\subsection{Execution}
\begin{tabularx}{\textwidth}{ | X | X | }
  \hline
  Relation & Event1 Log \\
  \hline
  Event 1 Execution begins & (\#1, Event 1, executionStart, ClientID) \\
  \hline
  Event 1 Execution fails & (\#2, Event 1, executionFail, ClientID) \\
  \hline
  Event 1 Execution success & (\#2, Event 1, executionSuccess, ClientID) \\
  \hline
\end{tabularx}

\subsection{Lock}
\begin{tabularx}{\textwidth}{|*{3}{>{\raggedright\arraybackslash}X|}}
  \hline
  Relation & Event1 Log & Event2 Log \\
  \hline
  Event 1 Locks Event 2 & (\#1, Event 1, lock, Event 2) & (\#1, Event 2, lockedBy, Event 1) \\
  \hline
  Event 2 Locks Event 1 & (\#1, Event 1, lockBy, Event 2) & (\#3, Event 2, lock, Event 1) \\
  \hline
\end{tabularx}

\subsection{Unlock}
\begin{tabularx}{\textwidth}{|*{3}{>{\raggedright\arraybackslash}X|}}
  \hline
  Relation & Event1 Log & Event2 Log \\
  \hline
  Event 1 Locks Event 2 & (\#1, Event 1, unlock, Event 2) & (\#1, Event 2, unlockedBy, Event 1) \\
  \hline
  Event 2 Locks Event 1 & (\#1, Event 1, unlockBy, Event 2) & (\#3, Event 2, unlock, Event 1) \\
  \hline
\end{tabularx}

\section{Happens Before Relations}

\begin{lstlisting}[breaklines=true]
    Y includedBy X       ->    X includes Y
    Y excludedBy X       ->    X excludes Y
    Y setPendingBy X     ->    X setsPending Y
    Y ConditionChecked X ->    X ConditionChecks Y
    Y LockedBy X         ->    X Lock Y
    Y UnlockedBy X       ->    X Unlock Y
    X Execution begins   ->    X Execution fails / Success
    X Execute Start      ->    X Locks Y
    X Locks Y            ->    X Include Y
    X Locks Y            ->    X Exclude Y
    X Locks Y            ->    X setPending Y
    X Locks Y            ->    X ConditionChecks Y
    X Locks Y            ->    X Unlock Y
    X Locks Y            ->    X Unlock Y     ->    X Execute Fail / Success
    Y Lockby X           ->    Y UnlockBy X   ->    Y LockBy "z"
\end{lstlisting}

\chapter{Algorithms}
The overall problem is: given a DCR graph with possible malicious nodes, for the events of the graph to reach consensus of a partial order of execution that has happened in that graph.

The algorithm consists of the following subalgorithms executed in that order:
\begin{itemize}
    \item Produce - which gathers the history.
    \item Elect - which checks if the majority of the workflow can accept the proposed history
    \item Simplify - which takes the history and creates an execution only history.
\end{itemize}

Correcness should be based on:
\begin{itemize}
    \item How many events can be corrupted.
    \item How well the algorithm can determine which events are corrupted.
    \item how well the algorithm can determine if any and how many events are corrupted.
\end{itemize}

%History is created by using Fetch-and-Stitch with validation in the stitching phase.

%The receiver of the first create history call should

%- create history ID
%- fetch history from neighbours
%    - each of these should:
%        - fetch
%        - stitch
%        - return
%- stitch
%- Call for a vote
%- Closure of event graph to an execution graph.
\section{Data Structure}\label{sec:datastructure}
\todo[inline]{Retfærdiggør valget af datastruktur.}
\begin{lstlisting}[breaklines=true]
type Node {
	Id: EventId + Local TimeStamp;
	Type: String or Relation type;
	... more information about the log entry ...
	Edges: Node.Id list;
}

type Graph {
	Nodes: Map<Node.Id, Node>	- evt. Map<Node.Id, (Node * Node.Id list)
}
\end{lstlisting}
% !TeX root = ../DCR-Consensus.tex
\section{Stitch}
\subsection{Analysis} % Hvad er problemet
\subsection{Implementation} % hvordan løses det
\subsection{Discussion} % bliver problemet løst
\subsection{Performance} % hvor godt løses det


% !TeX root = ../DCR-Consensus.tex
\section{Produce}
\subsection{Goal} % Hvad er problemet
The goal of the algortithm is, given a DCR graph, to fetch and merge the histories of each event in the graph into one history, such that the resulting history is partially ordered in accordance to the ruleset described in \autoref{chap:domainanalysis}. 

%Correcness should be based on:
%\begin{itemize}
%    \item How many nodes' history gets fetched in the workflow (higher is better).
%    \item How rendundant the data is (higher redundancy is better).
%    \item How well it handles cycles in the graph
%\end{itemize}

\subsection{Implementation} % hvordan løses det
The implementation of the algorithm starts at a single event in the workflow and is called recursively on the reachable events in the workflow. A reachable event in the workflow constitutes any event that has a relation from the current event in scope. \todo[inline]{Udvid beskrivelse til at inkludere løsning hvis event peger på nuværende event. Dette event er p.t. ikke "synligt".}


\begin{lstlisting}[breaklines=true]
 1. History is requested by `X` with `request trace` `T` and history ID: `HID`
    1. If (Lookup history for `HID`) is not empty
        - Return lookup history for `HID`
    - Add all relations to `wait for`
    - If `wait for` is empty
        - Create history
        - Return
    - For each node `n` in `T`
        - if `n` is in `wait for`
            - remove `n` from `wait for`
    - Create `T'` by appending own ID to `T`
    - If `wait for` is empty
        - Cyclic case: Return empty set -> maybe return local set
    - Ask all nodes in `wait for` for their history with `T'`
    - Stitch own history with answers
    - Return "new" history
\end{lstlisting}

\subsection{Discussion} % bliver problemet løst
\subsection{Performance} % hvor godt løses det
% !TeX root = ../DCR-Consensus.tex
% !TeX spellcheck = en_GB
\section{Elect}
\subsection{Goal} % Hvad er problemet
The goal of the algorithm is, given a partially ordered history, to reach consensus upon the history between the participating events of the history.

\subsection{Implementation}
Since the partially ordered history includes URIs to all the events of the system, the initiator of the history call can contact all events who have contributed to the history. By sending the history to all the contributing events, each of them can send back an Accept or Reject call, thereby having an election on the created history. Then according to the regular rules of distributed system elections, if the majority accepts the proposed value, and no more than a third of the events are byzantine the election can be accepted, and returned.
\todo[inline]{Source: An Optimal Probabilistic Protocol For Synchronous Byzantine Agreement - Pesech Feldman \& Silvio Micali - http://bit.ly/1Q30qAC} 

\subsection{Discussion} % bliver problemet løst
\subsection{Performance}
Requires \texttt{2N} calls where \texttt{N} is the number of events in the system.


% !TeX root = ../DCR-Consensus.tex
\section{Simplify}
\subsection{Goal} % Hvad er problemet
The goal of the algorithm is, given a partially ordered history, to create an order of execution, where execution is a particular form of log.
\todo{maybe do the problem description polymorphic}

\subsection{Implementation} % hvordan løses det
\begin{algorithmic}
	\State \Comment{Find start nodes}
	\ForAll{node in graph}
	    \If{node has no incoming edge}
		    \State add to startFromSet
		\EndIf
	\EndFor
	
	\State \Comment{Create transitive clousures}
	\ForAll{node1 in startFromSet}
	    \ForAll{edge to node2}
	        \If {node2 is execution log}
	            \State add edge from node1 to node2
	            \State add node2 to startFromSet
	            \State break
	        \EndIf
	    \EndFor
	\EndFor
	
	\State \Comment{Remove all non execution logs}
	\ForAll{node in graph}
	    \If {not execution log}
	        \State remove node from graph
	    \EndIf
	\EndFor
	      
	\State \Comment{Transitive reduction}
	\ForAll{node1 in graph}
	  \ForAll{node2 in graph}
	    \If{edge between node1 and node2 exists}
	      \ForAll{node3 in graph}
	        \If{edge between node1 and node3 exists}
	          \State remove node from node1 to node3
	        \EndIf
	      \EndFor
	    \EndIf
	  \EndFor
	\EndFor
\end{algorithmic}
\subsection{Discussion} % bliver problemet løst
\subsection{Performance} % hvor godt løses det


