% !TeX root = ../DistributedConsensus.tex
% !TeX spellcheck = en_GB
\chapter{Introduction}
	Dynamic Condition Response Graphs \todo{add citation to a DCR article} (“DCR Graphs”) are developed principally by Thomas Hildebrandt at the IT University of Copenhagen, in collaboration with ResultMaker and later Exformatics A/S. DCR graphs are used to model workflows which represents work processes.
	
	\newpar DCR graphs are used by companies to ensure that bussiness procedures are executed according to business rules. At a given time a business might want to see what has happened in a given DCR graph. We have defined this information as being the \textit{history} of the DCR graph. The history can contain interesting information for the business, as it might reveal which path has lead to the current state of the workflow. Furthermore, if a DCR graph is used to represent a case of a customer, and if multiple case workers are working on that case, and do not have a total overview of it, a history can provide that overview. Finally, for DCR graphs shared between two or more companies, one of the participating companies might want to ensure that the other companies have followed the rules of the DCR graph.
	
	\newpar The Events of a DCR graph can be distributed over a network allowing better scalability, reliability as well as allowing multiple companies to work together, by hosting and handling different events themselves. 
	
	\newpar In a distributed DCR graph finding a history or order of execution can be difficult, due to the fact that no single event has an overview of the entire workflow. Logs can be split among several events, the clocks of events are not necessarily synchronized, and events may emit erroneous logs. The events have to reach consensus on the history given these challenges. Even though there are challenges, DCR graphs provide information that can be used to relate logs with one another, which is especially helpful when logs of two or more events need to be combined into a history.
	
	\newpar By researching and applying distributed system theories and algorithms, as well as the rules of DCR Graphs, we will try to solve the problem of generating and reaching consensus of histories of DCR graphs. 