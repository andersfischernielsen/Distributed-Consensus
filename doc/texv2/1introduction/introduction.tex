% !TeX root = ../DistributedConsensus.tex
% !TeX spellcheck = en_GB
\chapter{Introduction}
	Dynamic Condition Response Graphs (“DCR Graphs”) are developed principally by Thomas Hildebrandt at the IT University of Copenhagen, in collaboration with ResultMaker and later Exformatics A/S. They are used to represent workflows, where nodes represent events, and edges between these nodes represent one of four different relations: condition, response, inclusion and exclusion. 
	
	\newpar The Events of a DCR graph can be distributed over a network allowing better scalability, reliability as well as allow multiple companies to work together, by hosting and handling different events themselves. 
	
	\newpar DCR graphs are used by companies to ensure that bussiness procedures are executed according to business rules. At a given time a business might want to see what has happened in a given DCR graph, we have defined this information as being the \textit{history} of the DCR graph. The history can contain interesting information for the business, as it might reveal which path in the workflow has lead to the current state. Furthermore if a DCR graph can be used to represents a case of a customer, and if multiple case workers are working on that case, and do not have a total overview of it, a history can provide that overview. Lastly for DCR graphs distributed among two or more companies, one of the participating companies might want to make sure that the other companies have done their job proberly.
	
	\newpar In a distributed DCR graph finding a history or log of a given order of execution can be difficult, due to the fact that no single node has the overview of the entire workflow. Logs can be split among several nodes, timestamps of logs do not necessarily correspond among nodes and nodes can emit erroneous logs. The evens have to reach consensus on the history given these challenges. Even though there are challenges, DCR graphs provide information that can be used to relate logs with eachother, which is especially helpful when histories of two or more events needs to be merged.
	
	\newpar By researching and applying distributed system theories and algorithms, as well as the rules of DCR Graphs, we will try to solve the problem of generating the history of a DCR graph. 