% !TeX root = ../DistributedConsensus.tex
% !TeX spellcheck = en_GB
\section{Problem Definition}
	\textit{Given a distributed DCR graph, the purpose of this project is to find an algorithm which uses the execution information of the individual events to find an order of execution where the information of these events are part of the result. Furthermore, the algorithm should be able to identify and remove the histories of malicious events from the order of execution. Correctly functioning events should be able to agree upon the resulting order of execution.}
	
	\vspace{0.4cm}
	
	\newpar
	This report has the following structure:
	
	\newpar
	Chapter \ref{chap:background} describes the previous work that this project is based upon. This includes distributed DCR graphs, graph theory, as well as distributed system concepts, such as consensus algorithms, serial equivalence and ordering of events.
		
	\newpar
	Chapter \ref{chap:representing-a-history} introduces the terms action and history and defines the representation of a history and an action.
	Furthermore, it examines why a local history of a single event must be in a strict total order and why this property is necessary when finding ordered history of multiple events. 
		
	\newpar
	Chapter \ref{chap:connecting-histories} examines how it is possible to exploit the rules of DCR graphs with no malicious events when merging histories of multiple events into a single history and explains why the result is in strict partial order.
	
	\newpar
	Chapter \ref{chap:consensusindcr} introduces malicious events in DCR graphs and what such events can do to a history. Furthermore it defines a valid history and describes how it is possible, both locally and globally, to validate histories using the constraints of DCR graphs, serial equivalence, and ordering of events.
	
    \newpar
    Chapter \ref{chap:order-of-execution} examines how it is possible to find an order of execution from a global history, and to prove the correctness of that order of execution it is shown that the participating events of the history can agree on the result.
		
	