% !TeX root = ../DistributedConsensus.tex
% !TeX spellcheck = en_GB
\section{Problem Definition} % Husk overgang med amerikansk lærebogsafsnit
	\textit{Given a distributed DCR graph, the purpose of this project is to use the individual events' histories to extract an order of execution, where each history is part of the resulting order of execution. Furthermore, the algorithm should be able to identify and remove histories from malicious events. The correctly functioning events should be able to agree upon the resulting order of execution}
	
	OR
	
	\textit{Given a history of events that have happened in a distributed implementation of DCR Graphs, the purpose of this project is to extract an order of execution from the given history. Furthermore, the order of execution should be validated in order to check for errors \todo{errors? Skal det uddybes?} in the history.}
	
	\vspace{0.6cm}
	
	\newpar
	This report has the following structure:
	
	\newpar
	In \autoref{chap:background} theory that this project is based upon is described. In this chapter distributed DCR graphs, distributed system concepts, such as consensus algorithms, serially equivalence and ordering and finally aspects of graph theory are detailed.
	The found solution to the problem definition is based upon concepts in this theory.
		
	\newpar
	In \autoref{chap:representing-a-history} the representation of a history and associated history action is defined in a simplified version of the problem domain. \todo{Definition i stedet for domain?}
	Furthermore, it examines why a local history of a single event must be totally ordered and describes how it is possible to extract an order of execution from a totally ordered history.
		
	\newpar
	In \autoref{chap:connecting-histories} the problem domain is extended by adding neighbouring events to a single event in a workflow, and defines the merging of logs across multiple events into one consistent history. Furthermore, it describes how it is possible to simplify a graph of logs into an order of execution.
	
	\newpar
	In \autoref{chap:consensusindcr} describes the validation of local and global histories across multiple events by validating history actions across events and using DCR graph requirements. It also defines how to determine when a history can be produced and, if possible, which events have inconsistent history data.
	
	\newpar
	In \autoref{chap:previous-attempts} attempts made at solving the problem during this project and the shortcomings of these attempt are described. Furthermore, it is described why the described solution in this report fulfills the shortcomings of these previous attempts. 
		
	