% !TeX root = ../DistributedConsensus.tex
% !TeX spellcheck = en_GB
\chapter*{Conclusion}\addcontentsline{toc}{chapter}{Conclusion}
This report has examined how it is possible to find a global order of execution from a run of a distributed DCR graph. Several algorithms are used in order to find this order of execution. By storing executions as actions performed by events as a totally ordered history on the individual events and using Lamport's logical clocks as timestamps to establish happens-before relations, it is possible to merge local histories together.

A combined history can be simplified to represent only the order of executions by collapsing actions of executions together to form single entities with the same happens-before relations, and finally using transitive reduction to find the minimum equivalent graph of the order of execution.

To reach distributed consensus on the order of execution, each event can examine the resulting order of execution and confirm that the local order execution of itself is preserved in the global order.

The observability and identifiability of malicious behaviour of processes in the DCR graph hosting events depend on both the type of cheating and the structure of the DCR graph. By applying validations on individual histories of events, pair validations between histories of two events and simulating the order of execution in the DCR graph, it is possible, in most cases, to observe malicious behaviour. If the amount of interconnectivity of the DCR graph is sparse, and malicious nodes are interconnected, then it is difficult to observe if some kind of cheating has happened at all.

Therefore it is up to the creators of workflows to create these with such structures and distribution of events across processes that all kinds of cheating are observable.

The time complexity of most algorithms have been analysed, and both from the analysis and from running the implementation it is clear that the transitive reduction algorithm used, is dominant factor in terms of computation time.

\newpar The transitive closure and the produce algorithm are examples of time consuming approaches to the problem that did not lead to sufficient results, satisfactory time complexities, or abilities to handle malicious processes. These algorithms are included as part of the report, to help the argumentation that the chosen algorithms are better approaches to solving the problem, given the distributed DCR graphs implementation which is the basis of the project.

\newpar Overall, algorithms have been described that solve the problem. Multiple areas could be extended upon in future projects, for example how malicious process are handled if they are identified, how much that can be said about a history if malicious behaviour is observed, with which structures of DCR graphs one could prevent malicious behaviour completely, if trust between processes could help in the identification of cheaters, in what cases it is impossible to find an order of execution at all, and finally, finding a peer-to-peer based gathering algorithm that would be able to handle malicious events.